\documentclass[11pt,fleqn,,a4paper,twoside,openright]{book}
%%%%%%%%%%%%%%%%%%%%%%%%%%%%%%%%%%%%%%%%%
% IHC Latex standard
% Structural Definitions File
% Version 1.0 (5/10/15)
%
% Original author:
% Jelle Spijker (j.spijker@ihcmerwede.com)
%%%%%%%%%%%%%%%%%%%%%%%%%%%%%%%%%%%%%%%%%

%----------------------------------------------------------------------------------------
% VARIOUS REQUIRED PACKAGES AND CONFIGURATIONS 
%----------------------------------------------------------------------------------------

\usepackage[top=3.5cm,bottom=3.75cm,left=3cm,headsep=10pt,a4paper]{geometry} % page margins

\usepackage{graphicx} % including pictures
\graphicspath{{Pictures/}}

\usepackage{tikz} %drawing custom shapes
\usetikzlibrary{shapes, arrows}

\usepackage[english]{babel}

\usepackage{enumitem}

\usepackage{lipsum}

\usepackage[nochapter,tocentry]{vhistory}

%----------------------------------------------------------------------------------------
%	Header
%----------------------------------------------------------------------------------------

\usepackage{fancyhdr}
\pagestyle{fancy}
\usepackage[odd]{emptypage}

\usepackage{etoolbox}
\patchcmd{\chapter}{\thispagestyle{plain}}{\thispagestyle{fancy}}{}{}

\fancyhead{}
\fancyhead[LO,LE]{\includegraphics[width=3cm]{Logo.png}}
\fancyfoot[LO,LE]{\includegraphics[width=6cm]{Mission.png}}
\fancyfoot[RO, RE]{\includegraphics[width=7cm]{Copyright.png}}
%
%\makeatletter
%\def\cleardoublepage{\clearpage\if@twoside \ifodd\c@page\else
%	\hbox{}
%	\thispagestyle{plain}
%	\newpage
%	\if@twocolumn\hbox{}\newpage\fi\fi\fi}
%\makeatother \clearpage{\pagestyle{plain}\cleardoublepage}

%----------------------------------------------------------------------------------------
%	Colors
%----------------------------------------------------------------------------------------

\usepackage{xcolor} % Required for specifying colors by name
\definecolor{ocre}{RGB}{243,0,0} 

%----------------------------------------------------------------------------------------
%	Fonts
%----------------------------------------------------------------------------------------
\usepackage{helvet}
\renewcommand{\familydefault}{\sfdefault}

\usepackage{titlesec}
\usepackage{etoolbox}
\usepackage{lipsum}

\setcounter{secnumdepth}{5}
\renewcommand\thesection{\arabic{section}}

%----------------------------------------------------------------------------------------
%	Chapter format
%----------------------------------------------------------------------------------------

\usepackage{titlesec}
\titleformat{\chapter}[display]
{\normalfont\huge\bfseries}{\MakeUppercase{\chaptertitlename}\ \thechapter}{5pt}{\Huge}
\titlespacing{\chapter}{0pt}{*0}{*5}

%----------------------------------------------------------------------------------------
%	REMARK ENVIRONMENT
%----------------------------------------------------------------------------------------

\begin{document}
	
	\begin{titlepage}
		\centering
		\includegraphics[width=0.35\textwidth]{Logo.png}\par\vspace{1cm}
		{\scshape\LARGE Royal IHC \par}
		\vspace{1cm}
		{\huge\bfseries Proposal P\&ID toolbox\par}
		\vspace{0.7cm}
		{\LARGE\itshape for Excel\par}
		\vspace{2cm}
		written by\par
		{\Large\itshape Jelle Spijker\par}
		\vfill
		supervised by\par
		Andr\'{e} Trouwborst
		
		\vfill
		
		% Bottom of the page
		{\large \today\par}
	\end{titlepage}
	
	\hspace{2cm}
	\tableofcontents
	
	\chapter{Introduction}\label{chap:Introduction}
	Current P\&ID calculations are done in Excel with "copy-paste" sheets from old projects. This leaves a lot of room for interpretation. Whilst basic calculations used during the design and verification of P\&ID's are based on non-changing thermo- and fluid dynamical principles. It is proposed to  make an toolbox which integrates with Excel 2010 and allows users to create uniform calculation sheets. I
	
	\section{Document history}
	\begin{versionhistory}
		\vhEntry{1.0}{19-10-2015}{Jelle Spijker}{Initial released document}	
	\end{versionhistory}
	
	\section{Intended audience}\label{sec:Intended audience}
	Decision makers, ICT professionals and P\&ID specialists
	
	\section{Prerequisites}\label{sec:Prerequisites}
	\begin{itemize}
		\item none
	\end{itemize}
	
	\section{Objective}\label{sec:Objective}
	To illustrate the possible workings of a P\&ID toolbox and convey the needed resources with different scenarios.
	
	\section{Document structure}\label{sec:Document structure}
	In chapter \ref{chap:Current working sheets} the current situation is illustrated
	
	\chapter{Current working sheets}\label{chap:Current working sheets}
	 
	

\end{document}